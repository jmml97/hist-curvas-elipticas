%$ xelatex --shell-escape $0
% Maria Eichlseder, June 2016
% https://www.iacr.org/authors/tikz/
\documentclass[crop,tikz]{standalone}

\usepackage[no-math]{fontspec}
\setmainfont{Libertinus Serif}
\setsansfont{Libertinus Sans}
\setmonofont{Libertinus Mono}

\usepackage[math-style=TeX]{unicode-math}
\setmathfont{Libertinus Math}

\usepackage{tikz}
\usepackage{amsfonts}
\usetikzlibrary{intersections,decorations.markings}

\usepackage{pifont}


\newcommand{\plotcurve}[3][thick, every plot/.style={smooth}]{
  % plot curve y^2 = x^3 + a x + b in range [-3,3]^2
  % parameter 1 (optional): style options for curve (color, etc)
  % parameter 2: curve parameter a
  % parameter 3: curve parameter b
  \draw[gray] (-3,-3) rectangle (3,3);
  \draw[->,>=latex,gray] (-3,0) -- (3,0);
  \draw[->,>=latex,gray] (0,-3) -- (0,3);
  \draw[name path=curve, #1] plot[id=curve#2#3, raw gnuplot] function {
    f(x,y) = y**2 - x**3 - #2*x - #3;
    set xrange [-3:3];
    set yrange [-3:3];
    set view 0,0;
    set isosample 50,50;
    set cont base;
    set cntrparam levels incre 0,0.1,0;
    unset surface;
    splot f(x,y);
  };
}

% For an explanation of the tangent coordinate system, check http://tex.stackexchange.com/a/25940 
\tikzset{
  tangent/.style={
    decoration={markings, mark=at position #1 with {
      \coordinate (tangent point-\pgfkeysvalueof{/pgf/decoration/mark info/sequence number}) at (0pt,0pt);
      \coordinate (tangent unit vector-\pgfkeysvalueof{/pgf/decoration/mark info/sequence number}) at (1,0pt);
      \coordinate (tangent orthogonal unit vector-\pgfkeysvalueof{/pgf/decoration/mark info/sequence number}) at (0pt,1);
    }},
    postaction=decorate
  },
  use tangent/.style={
    shift=(tangent point-#1),
    x=(tangent unit vector-#1),
    y=(tangent orthogonal unit vector-#1)
  },
  use tangent/.default=1
}


\begin{document}
  \sffamily
  \begin{tikzpicture}[scale=.5]

    \draw[gray] (-3,-3) rectangle (3,3);
	  \draw[->,>=latex,gray] (-3,0) -- (3,0);
	  \draw[->,>=latex,gray] (0,-3) -- (0,3);
	  \draw[name path=curve] plot[id=curve-22, raw gnuplot] function {
	    f(x,y) = y**2 - x**3 + 2*x - 2;
	    set xrange [-3:3];
	    set yrange [-3:3];
	    set view 0,0;
	    set isosample 50,50;
	    set cont base;
	    set cntrparam levels incre 0,0.1,0;
	    unset surface;
	    splot f(x,y);
	  };
	  \draw[tangent=0.265, thick] plot[id=curve-22bis, raw gnuplot] function {
	    f(x,y) = y**2 - x**3 + 2*x - 2;
	    set xrange [-3:3];
	    set yrange [-3:3];
	    set view 0,0;
	    set isosample 50,50;
	    set cont base;
	    set cntrparam levels incre 0,0.1,0;
	    unset surface;
	    splot f(x,y);
	  };
    \draw[thick, use tangent, name path=chord] (-.6,0) -- (1.5,0);
    \draw[name intersections={of=curve and chord}] (intersection-1) node {●} node[above] {$P$}
                                                    (intersection-2) node {●} coordinate[name=mPQ];
    \draw[dashed, semithick, name path=vertical] (mPQ) ++(0,1) -- +(0,-5.0);
    \draw[name intersections={of=curve and vertical}] (intersection-2) node {●} node[below left] {$2P$};
    \node[below, align=center] at (0,-4) {Duplicar $P+P$ \\ 
    %<<Regla de la tangente>>
    };
    
  \end{tikzpicture}

\end{document}
