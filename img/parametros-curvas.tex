%$ xelatex --shell-escape $0
% Maria Eichlseder, June 2016
% https://www.iacr.org/authors/tikz/
\documentclass[crop,tikz]{standalone}

\usepackage[no-math]{fontspec}
\setmainfont{Libertinus Serif}
\setsansfont{Libertinus Sans}
\setmonofont{Libertinus Mono}

\usepackage[math-style=TeX]{unicode-math}
\setmathfont{Libertinus Math}

\usepackage{tikz}
\usepackage{amsfonts}
\usetikzlibrary{intersections}


\newcommand{\plotcurve}[3][thick, every plot/.style={smooth}]{
  % plot curve y^2 = x^3 + a x + b in range [-3,3]^2
  % parameter 1 (optional): style options for curve (color, etc)
  % parameter 2: curve parameter a
  % parameter 3: curve parameter b
  \draw[gray] (-3,-3) rectangle (3,3);
  \draw[->,>=latex,gray] (-3,0) -- (3,0);
  \draw[->,>=latex,gray] (0,-3) -- (0,3);
  \draw[name path=curve, #1] plot[id=curve#2#3, raw gnuplot] function {
    f(x,y) = y**2 - x**3 - #2*x - #3;
    set xrange [-3:3];
    set yrange [-3:3];
    set view 0,0;
    set isosample 50,50;
    set cont base;
    set cntrparam levels incre 0,0.1,0;
    unset surface;
    splot f(x,y);
  };
}

\begin{document}
  \sffamily
  \begin{tikzpicture}
    \foreach \a in {-2,...,1} {
      \foreach \b in {-1,...,2} {
        \begin{scope}[xshift=1.75*\b cm,yshift=-1.75*\a cm,scale=.25]
          \plotcurve{\a}{\b}
        \end{scope}
      }
      \draw (-1.75*2+1,-1.75*\a) node[above,rotate=90] {$A=\a$};
    }
    \foreach \b in {-1,...,2} {
      \draw (1.75*\b,1.75*3-1) node[above] {$B=\b$};
    }
    \node[above] at (1.75/2,-2*1.75) {$y^2 = x^3 + A \cdot x + B$};
  \end{tikzpicture}
\end{document}