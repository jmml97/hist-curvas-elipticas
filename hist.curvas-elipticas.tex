\documentclass[
  a4paper,
  12pt,
  spanish,
]{scrartcl}

% Párrafos
\setlength{\parindent}{18pt}

%-------------------------------------------------------------------------------
%	PAQUETES
%-------------------------------------------------------------------------------

% Idioma

\usepackage[es-noindentfirst]{babel}

% Citas de texto en línea/bloque

\usepackage[autostyle]{csquotes}

% Matemáticas

\usepackage{amsmath, amsthm, amssymb}
\usepackage{mathtools}
\usepackage{commath}

% Fuentes personalizadas para utilizar con XeLaTeX o LuaLaTeX

\usepackage[no-math]{fontspec}
\setmainfont{Libertinus Serif}
\setsansfont{Libertinus Sans}
\setmonofont{Libertinus Mono}

\usepackage[math-style=TeX]{unicode-math}
\setmathfont{Libertinus Math}

% Configuración de microtype

\defaultfontfeatures{Ligatures=TeX,Numbers=Lining}
\usepackage[activate={true,nocompatibility},final,tracking=true,factor=1100,stretch=10,shrink=10]{microtype}
\SetTracking{encoding={*}, shape=sc}{0}

% Enlaces y colores

\usepackage{hyperref}
\usepackage[dvipsnames]{xcolor}
\definecolor{webgreen}{rgb}{0,0.5,0}
\hypersetup{
  colorlinks=true,
  citecolor=webgreen,
}

% Otros elementos de página

\usepackage{enumitem}
\setlist[enumerate]{leftmargin=*, itemsep=0pt}
\setlist[itemize]{leftmargin=*, itemsep=0pt}

\usepackage[labelfont=sc]{caption}

% Tikz

\usepackage{tikz}
\usetikzlibrary{babel}
\usepackage{float}

% Código

\usepackage{listings}
\lstset{
	basicstyle=\footnotesize\ttfamily,%
	breaklines=true,%
	captionpos=b,                    % sets the caption-position to bottom
  tabsize=2,	                   % sets default tabsize to 2 spaces
  frame=lines,
  numbers=left,
  stepnumber=1,
  aboveskip=12pt,
  showstringspaces=false,
}
\renewcommand{\lstlistingname}{Listado}

% Bibliografía

\usepackage[sorting=none, style=apa, isbn=true]{biblatex}
\addbibresource{bibliografia.bib}

% Lorem ipsum

\usepackage{blindtext}

% Márgenes
\usepackage[bottom=3.125cm, top=2.5cm, left=4.5cm, right=4.5cm, marginparwidth=70pt]{geometry}

% Fuentes

\usepackage{textcase}

\newfontfamily{\sacshape}{Libertinus Serif}[
  WordSpace={1.8},
  LetterSpace={18.0}
]

\newfontfamily{\slscshape}{Libertinus Serif}[
  WordSpace={1.8},
  LetterSpace={6.0}
]

\DeclareRobustCommand{\spacedallcaps}[1]{{\linespread{1.3}\sacshape\MakeTextUppercase{#1}}}% WordSpace=1.8
\DeclareRobustCommand{\spacedlowsmallcaps}[1]{{\slscshape\MakeTextLowercase{#1}}}% WordSpace=1.8

% Cabeceras de sección

\RedeclareSectionCommands[beforeskip=-3ex,
afterskip=2ex]{section,subsection,subsubsection}
%\addtokomafont{section}{\normalfont\large\spacedallcaps}
%\setkomafont{section}{\normalfont\large\scshape}
\RedeclareSectionCommand[beforeskip=-9ex, font=\normalfont\large\scshape, tocentryformat=\normalfont\scshape]{section}
\addtokomafont{subsection}{\normalfont\normalsize\itshape}
\RedeclareSectionCommand[beforeskip=-6ex,tocentryformat=\normalfont\itshape]{subsection}
\addtokomafont{subsubsection}{\normalfont}
\RedeclareSectionCommand[beforeskip=-4ex]{subsubsection}
\addtokomafont{paragraph}{\normalfont\itshape}
%-------------------------------------------------------------------------------
%	TÍTULO
%-------------------------------------------------------------------------------

\newcommand{\horrule}[1]{\rule{\linewidth}{#1}}

%-------------------------------------------------------------------------------
%	CONTENIDO
%-------------------------------------------------------------------------------

\begin{document}

\begin{titlepage}
  \vspace*{4cm}

  \begin{flushleft}
    \Huge
    \spacedallcaps{Curvas Elípticas en la Criptografía}
    \horrule{2pt}
  \end{flushleft}

  \vspace{2em}

  \begin{flushright}
    \large
    Sofía Almeida Bruno\\
    Antonio Coín Castro\\
    José María Martín Luque\vspace{1em}
  
    \textit{Historia de las Matemáticas}
  
    Grado en Matemáticas
  
    \textsc{Universidad de Granada}\vspace{1em}
  
    \today\vspace{.5em}
  \end{flushright}
\end{titlepage}

\newpage

{\hypersetup{hidelinks}
\tableofcontents
}

\newpage

\section{Introducción}

El objetivo principal de la criptografía es el de la transmisión de información confidencial a través de un canal inseguro.

\section{Curvas elípticas}

\subsection{Curvas elípticas sobre los números reales}

%Citas: \parencite{abedon-hyman-2003}

%Antonio.

%Definición, fórmula (simplificación?).

%Estructura de grupo. Sumar puntos. Dibujitos.

\subsection{Curvas elípticas sobre cuerpos finitos}

%Antonio.

%Fórmula concreta, analogía con el caso de R.

\section{Evolución de las curvas elípticas en la criptografía}

Referenciar a \parencite{singh_code_2003}, \parencite{thawte_history_2013}, \parencite{diffie_new_1976}.

Un criptosistema es una familia uniparamétrica \(\{S_K\}_{K \in \{K\}}\) de aplicaciones invertibles \[S_K: \{P\} \to \{C\},\] del conjunto de mensajes en claro, \(\{P\}\), al conjunto de mensajes cifrados, \(\{C\}\). 
El parámetro \(K\) se denomina \textit{clave} y el rango de posibles valores que puede tomar \(\{K\}\), \textit{espacio de claves}.

El objetivo a la hora de diseñar un criptosistema \(\{S_K\}\) ha de ser procurar que las operaciones de cifrado y descifrado resulten sencillas pero asegurarse que el análisis criptográfico por parte de terceras personas con el fin de interceptar y descifrar los mensajes cifrados sea lo suficientemente difícil.

\subsection{Criptografía moderna anterior a los años 80}

%Jose.

% - Algoritmo de diffie-hellman, el problema del logaritmo discreto.
% - Tamaño de las claves, número de parámetros, éxito.
% - Problemas existentes.

En los años 70 del siglo pasado se produjeron dos acontecimientos públicos que supusieron grandes avances en el desarrollo de la criptografía moderna.

En primer lugar, en 1973 la Oficina Nacional de Normas (NBS por sus siglas en inglés, \textit{National Bureau of Standards}), dependiente del Departamento de Comercio del Gobierno Federal de los Estados Unidos, organizó un concurso público para el diseño de un algoritmo de cifrado que pudiese ser adoptado como estándar por parte de dicho Gobierno. 
El 17 de marzo de 1975 se publicó el primer borrador del \textit{Data Encryption Standard} (DES), sistema de cifrado propuesto por un grupo de investigación de IBM.
Tras recibir varias modificaciones con la ayuda de la Agencia de Seguridad Nacional (NSA por sus siglas en inglés, \textit{National Security Agency}) la versión final del algoritmo fue aprobada y publicada por la NBS en noviembre de 1976 y se convirtió en el primer estándar aprobado por una agencia del Gobierno Federal de los Estados Unidos. 
El hecho de que la NSA colaborase en el diseño del algoritmo --modificando la propuesta original de IBM-- suscitó las sospechas de numerosos investigadores, quienes creían que la NSA había introducido una \textit{puerta trasera} en el algoritmo. 
Finalmente, en los años 90 se llegó a la conclusión de que los cambios aportados por la NSA resultaron ser mejoras. %TODO: citation needed
Otra de las críticas que suscitó fue el tamaño elegido para los bloques.
En 2002 fue reemplazado por el algoritmo AES (\textit{Advanced Encryption Standard}), aprobado como estándar por el Instituto Nacional de Estándares y Tecnología (NIST por sus siglas en inglés, \textit{National Institute of Standards and Technology}).

\subsubsection{Protocolo de Diffie y Hellman}

El segundo avance significativo, y el que verdaderamente es relevante para nuestro estudio, fue la publicación en 1976 del artículo \textit{New Directions in Cryptography} \parencite{diffie_new_1976} en el que los autores introducen por primera vez la idea del cifrado de clave pública. 
Hasta el momento, todo algoritmo de cifrado necesitaba una clave secreta, compartida únicamente entre el emisor y el receptor. 
La dificultad para llevar a cabo la comunicación cifrada radicaba entonces en la transmisión de dicha clave de forma segura. 
Diffie y Hellman propusieron un criptosistema asimétrico que utiliza una pareja de claves: una pública y otra privada, eliminando así el obstáculo que suponía la transmisión de la clave.

La técnica se basa en la aparente dificultad de calcular logaritmos en un cuerpo finito de \(q\) elementos, denotado \(\mathbb F_q\), siendo \(q\) un número primo. 
Sea \[Y = \alpha^X \mod q, \qquad \text{para } 1 \leq X \leq q - 1,\] donde \(\alpha\) es fijo y es una \textit{raíz primitiva módulo} \(q\), es decir, todo número coprimo con \(q\) es congruente a una potencia de \(\alpha\) módulo \(q\). 
Al elemento \(X\) se le llama \textit{logaritmo de} \(Y\) \textit{de base} \(\alpha\) y \textit{módulo} \(q\): \[X = \log_{\alpha} Y \mod q, \qquad \text{para } 1 \leq X \leq q - 1.\]
El cálculo de \(Y\) a partir de \(X\) es sencillo, necesitándose para ello como mucho \(2 \cdot \log_2 q\) multiplicaciones utilizando el algoritmo de \textit{exponenciación binaria}. Por ejemplo, para \(X = 18\), \[Y = \alpha^{18} = \cramped{(((\alpha^2)^2)^2)^2} \cdot \cramped{\alpha^2}.\]
Sin embargo, la operación inversa de calcular \(X\) a partir de \(Y\) es mucho más compleja, y en ello radica la seguridad de este sistema.

A la hora de establecer la comunicación cifrada cada usuario genera un número aleatorio \(X_i\) de forma independiente dentro del rango de enteros \(\{1, 2, \dots, q - 1\}\). 
Dicho número se guarda en secreto, pero se hace público el número \[Y_i = \alpha^{X_i} \mod q\] junto al nombre y dirección que permiten identificar al usuario. 
Cuando los usuarios \(i\) y \(j\) quieren comunicarse de forma segura utilizan la clave \(K_{ij} = \cramped{\alpha^{X_iX_j}} \mod q\). 
El usuario \(i\) obtiene la clave \(K_{ij}\) a partir del número público \(Y_j\) realizando las operaciones \begin{align*}
  K_{ij} &= Y_j^{X_i} \mod q \\
    &= (\alpha^{X_j})^{X_i} \mod q \\
    &= \alpha^{X_jX_i} \mod q \\
    &= \alpha^{X_iX_j} \mod q.
\end{align*}
El usuario \(j\) obtiene la clave de forma análoga, calculando en su caso \[K_{ij} = Y_i^{X_j} \mod q.\]

\subsubsection{Problema de Diffie-Hellman}

Para que una tercera persona pudiese interceptar la comunicación tendría que obtener la clave \(K_{ij}\) a partir de \(Y_i\) e \(Y_j\), para lo que tendría que calcular, por ejemplo, \[K_{ij} = Y_i^{\left(log_{\alpha} Y_j\right)} \mod q.\] 
Este problema se denomina \textit{problema de Diffie-Hellman}. 
En su artículo los autores conjeturan que la resolución de dicho problema es equivalente a resolver el \textit{problema del logaritmo discreto} para alguno de los valores \(Y_i\) o \(Y_j\). Este problema puede enunciarse como: \begin{displayquote}
  Sea \(G\) un grupo y \(\langle g \rangle\) el subgrupo cíclico generado por \(g\). Dado \(g \in G\) y \(a \in \langle g \rangle\), encuentra \(x\) tal que \(g^x = a\).
\end{displayquote}
En 1990 Bert den Boer afirmó que para ciertos primos, el problema de Diffie-Hellman es tan difícil de resolver como el del logaritmo discreto \parencite{goos_diffie-hellman_1990}, tal y como aventuraron Diffie y Hellman 14 años antes.
Hasta ahora no se ha encontrado ningún algoritmo que resuelva este problema en tiempo polinómico. %TODO citation needed, aclarar qué es tiempo polinómico de forma rápida, sencilla y para toda la familia 


\subsection{Primera aparición de las curvas elípticas en criptografía}
% Enlazar con el apartado anterior

%1984. Algoritmo de Lenstra
La primera aparición de las curvas elípticas en criptografía es en el algoritmo de factorización de Lenstra, propuesto en 1984. En este momento se comenzó a buscar usos criptográficos para muchas herramientas matemáticas que nunca habían sido utilizadas con este propósito.

%¿Explicar?

%1987. Problema del logaritmo discreto con curvas elípticas. N. Koblitz y V. Miller (de forma independiente)
Al año siguiente, N. Koblitz y V. Miller proponen, de forma independiente, un uso diferente para las curvas elípticas en criptografía. Su idea fue usar un criptosistema como el de Diffie y Hellman utilizando, en vez del grupo multiplicativo sobre un cuerpo finito, el grupo de puntos de una curva elíptica sobre un cuerpo finito. 

%.... (paper)

%Tamaño de las claves, número de parámetros, éxito.

%Problemas existentes

\subsection{<<Paradigm shift>>}

%Sofi.

%Pairing-based cryptography

\subsection{Estado actual y algoritmos utilizados}

%Antonio.

\section{Usos futuros}

%Jose.

%Quantum. https://www.esat.kuleuven.be/cosic/elliptic-curves-are-quantum-dead-long-live-elliptic-curves/

%-------------------------------------------------------------------------------
%	BIBLIOGRAFÍA
%-------------------------------------------------------------------------------

\newpage
\printbibliography

\end{document}
