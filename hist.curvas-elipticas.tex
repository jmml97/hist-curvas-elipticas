\documentclass[
  a4paper,
  12pt,
  spanish,
]{scrartcl}

% Párrafos
\setlength{\parindent}{18pt}

%-------------------------------------------------------------------------------
%	PAQUETES
%-------------------------------------------------------------------------------

% Idioma

\usepackage[es-noindentfirst]{babel}

% Matemáticas

\usepackage{amsmath, amsthm, amssymb}
\usepackage{mathtools}
\usepackage{commath}

% Fuentes personalizadas para utilizar con XeLaTeX o LuaLaTeX

\usepackage[no-math]{fontspec}
\setmainfont{Libertinus Serif}
\setsansfont{Libertinus Sans}
\setmonofont{Libertinus Mono}

\usepackage[math-style=TeX]{unicode-math}
\setmathfont{Libertinus Math}

% Configuración de microtype

\defaultfontfeatures{Ligatures=TeX,Numbers=Lining}
\usepackage[activate={true,nocompatibility},final,tracking=true,factor=1100,stretch=10,shrink=10]{microtype}
\SetTracking{encoding={*}, shape=sc}{0}

% Enlaces y colores

\usepackage{hyperref}
\usepackage[dvipsnames]{xcolor}
\definecolor{webgreen}{rgb}{0,0.5,0}
\hypersetup{
  colorlinks=true,
  citecolor=webgreen,
}

% Otros elementos de página

\usepackage{enumitem}
\setlist[enumerate]{leftmargin=*, itemsep=0pt}
\setlist[itemize]{leftmargin=*, itemsep=0pt}

\usepackage[labelfont=sc]{caption}

% Tikz

\usepackage{tikz}
\usetikzlibrary{babel}
\usepackage{float}

% Código

\usepackage{listings}
\lstset{
	basicstyle=\footnotesize\ttfamily,%
	breaklines=true,%
	captionpos=b,                    % sets the caption-position to bottom
  tabsize=2,	                   % sets default tabsize to 2 spaces
  frame=lines,
  numbers=left,
  stepnumber=1,
  aboveskip=12pt,
  showstringspaces=false,
}
\renewcommand{\lstlistingname}{Listado}

% Bibliografía

\usepackage[sorting=none, style=apa, isbn=true]{biblatex}
\addbibresource{bibliografia.bib}

% Lorem ipsum

\usepackage{blindtext}

% Márgenes
\usepackage[bottom=3.125cm, top=2.5cm, left=4.5cm, right=4.5cm, marginparwidth=70pt]{geometry}

% Fuentes

\usepackage{textcase}

\newfontfamily{\sacshape}{Libertinus Serif}[
  WordSpace={1.8},
  LetterSpace={18.0}
]

\newfontfamily{\slscshape}{Libertinus Serif}[
  WordSpace={1.8},
  LetterSpace={6.0}
]

\DeclareRobustCommand{\spacedallcaps}[1]{{\linespread{1.3}\sacshape\MakeTextUppercase{#1}}}% WordSpace=1.8
\DeclareRobustCommand{\spacedlowsmallcaps}[1]{{\slscshape\MakeTextLowercase{#1}}}% WordSpace=1.8

% Cabeceras de sección

\RedeclareSectionCommands[beforeskip=-3ex,
afterskip=2ex]{section,subsection,subsubsection}
%\addtokomafont{section}{\normalfont\large\spacedallcaps}
%\setkomafont{section}{\normalfont\large\scshape}
\RedeclareSectionCommand[beforeskip=-9ex, font=\normalfont\large\scshape, tocentryformat=\normalfont\scshape]{section}
\addtokomafont{subsection}{\normalfont\normalsize\itshape}
\RedeclareSectionCommand[beforeskip=-6ex,tocentryformat=\normalfont\itshape]{subsection}
\addtokomafont{subsubsection}{\normalfont}
\RedeclareSectionCommand[beforeskip=-4ex]{subsubsection}
\addtokomafont{paragraph}{\normalfont\itshape}
%-------------------------------------------------------------------------------
%	TÍTULO
%-------------------------------------------------------------------------------

\newcommand{\horrule}[1]{\rule{\linewidth}{#1}}

%-------------------------------------------------------------------------------
%	CONTENIDO
%-------------------------------------------------------------------------------

\begin{document}

\begin{titlepage}
  \vspace*{4cm}

  \begin{flushleft}
    \Huge
    \spacedallcaps{Curvas Elípticas en la Criptografía}
    \horrule{2pt}
  \end{flushleft}

  \vspace{2em}

  \begin{flushright}
    \large
    Sofía Almeida Bruno\\
    Antonio Coín Castro\\
    José María Martín Luque\vspace{1em}
  
    \textit{Historia de las Matemáticas}
  
    Grado en Matemáticas
  
    \textsc{Universidad de Granada}\vspace{1em}
  
    \today\vspace{.5em}
  \end{flushright}
\end{titlepage}

\newpage

{\hypersetup{hidelinks}
\tableofcontents
}

\newpage

\section{Introducción}

\section{Curvas elípticas}

\subsection{Curvas elípticas sobre los números reales}

%Citas: \parencite{abedon-hyman-2003}

%Antonio.

%Definición, fórmula (simplificación?).

%Estructura de grupo. Sumar puntos. Dibujitos.

\subsection{Curvas elípticas sobre cuerpos finitos}

%Antonio.

%Fórmula concreta, analogía con el caso de R.

\section{Evolución de las curvas elípticas en la criptografía}

\subsection{Criptografía anterior a los años 80}

%Jose.

%Algoritmo de diffie-hellman, el problema del logaritmo discreto.

%Tamaño de las claves, número de parámetros, éxito.

%Problemas existentes.

\subsection{Primera aparición de las curvas elípticas en criptografía}

%Sofi.

%1987. Algoritmo de factorización de H.W Lenstra

%1987. Problema del logaritmo discreto con curvas elípticas. N. Koblitz y V. Miller (de forma independiente)

%.... (paper)

%Tamaño de las claves, número de parámetros, éxito.

%Problemas existentes

\subsection{<<Paradigm shift>>}

%Sofi.

%Pairing-based cryptography

\subsection{Estado actual y algoritmos utilizados}

%Antonio.

\section{Usos futuros}

%Jose.

%Quantum. https://www.esat.kuleuven.be/cosic/elliptic-curves-are-quantum-dead-long-live-elliptic-curves/

%-------------------------------------------------------------------------------
%	BIBLIOGRAFÍA
%-------------------------------------------------------------------------------

\newpage
\printbibliography

\end{document}
